%-----------------------------------------------------------------------------%
\chapter{\babDua}
%-----------------------------------------------------------------------------%
\todo{tambahkan kata-kata pengantar bab 2 disini}

%-----------------------------------------------------------------------------%
\section{\latex~Secara Singkat}
%-----------------------------------------------------------------------------%
Apa itu \LaTeX \\
\begin{tabular}{| p{13cm} |}
	\hline 
	\\
	LaTeX is a family of programs designed to produce publication-quality 
	typeset documents. It is particularly strong when working with 
	mathematical symbols. \\	
	The history of LaTeX begins with a program called TEX. In 1978, a 
	computer scientist by the name of Donald Knuth grew frustrated with the 
	mistakes that his publishers made in typesetting his work. He decided 
	to create a typesetting program that everyone could easily use to 
	typeset documents, particularly those that include formulae, and made 
	it freely available. The result is TEX. \\	
	Knuth's product is an immensely powerful program, but one that does 
	focus very much on small details. A mathematician and computer 
	scientist by the name of Leslie Lamport wrote a variant of TEX called 
	LaTeX that focuses on document structure rather than such details. \\
	\\
	\hline
\end{tabular}

\vspace*{0.8cm}

Dokumen \latex~sangat mudah, seperti halnya membuat dokumen teks biasa. Ada 
beberapa perintah yang diawali dengan tanda '\bslash'. 
Seperti perintah \bslash\bslash~yang digunakan untuk memberi baris baru. 
Perintah tersebut juga sama dengan perintah \bslash newline. 
Pada bagian ini akan sedikit dijelaskan cara manipulasi teks dan 
perintah-perintah \latex~yang mungkin akan sering digunakan. 
Jika ingin belajar hal-hal dasar mengenai \latex, silahkan kunjungi: 

\begin{itemize}
	\item \url{http://frodo.elon.edu/tutorial/tutorial/}, atau
	\item \url{http://www.maths.tcd.ie/~dwilkins/LaTeXPrimer/}
\end{itemize}


%-----------------------------------------------------------------------------%
\section{\latex~Kompiler dan IDE}
%-----------------------------------------------------------------------------%
Agar dapat menggunakan \latex~(pada konteks hanya sebagai pengguna), Anda 
tidak perlu banyak tahu mengenai hal-hal didalamnya. 
Seperti halnya pembuatan dokumen secara visual (contohnya Open Office (OO) 
Writer), Anda dapat menggunakan \latex~dengan cara yang sama. 
Orang-orang yang menggunakan \latex~relatif lebih teliti dan terstruktur 
mengenai cara penulisan yang dia gunakan, \latex~memaksa Anda untuk seperti 
itu.  

Kembali pada bahasan utama, untuk mencoba \latex~Anda cukup mendownload 
kompiler dan IDE. Saya menyarankan menggunakan Texlive dan Texmaker. 
Texlive dapat didownload dari \url{http://www.tug.org/texlive/}. 
Sedangkan Texmaker dapat didownload dari 
\url{http://www.xm1math.net/texmaker/}. 
Untuk pertama kali, coba buka berkas thesis.tex dalam template yang Anda miliki 
pada Texmaker. 
Dokumen ini adalah dokumen utama. 
Tekan F6 (PDFLaTeX) dan Texmaker akan mengkompilasi berkas tersebut menjadi 
berkas PDF. 
Jika tidak bisa, pastikan Anda sudah menginstall Texlive. 
Buka berkas tersebut dengan menekan F7. 
Hasilnya adalah sebuah dokumen yang sama seperti dokumen yang Anda baca saat 
ini. 

%-----------------------------------------------------------------------------%
\section{thesis.tex}
%-----------------------------------------------------------------------------%
Berkas ini berisi seluruh berkas Latex yang dibaca, jadi bisa dikatakan sebagai 
berkas utama. Dari berkas ini kita dapat mengatur bab apa saja yang ingin 
kita tampilkan dalam dokumen.


%-----------------------------------------------------------------------------%
\section{laporan\_setting.tex}
%-----------------------------------------------------------------------------%
Berkas ini berguna untuk mempermudah pembuatan beberapa template standar. 
Anda diminta untuk menuliskan judul laporan, nama, npm, dan hal-hal lain yang 
dibutuhkan untuk pembuatan template. 


%-----------------------------------------------------------------------------%
\section{istilah.tex}
%-----------------------------------------------------------------------------%
Berkas istilah digunakan untuk mencatat istilah-istilah yang digunakan. 
Fungsinya hanya untuk memudahkan penulisan.
Pada beberapa kasus, ada kata-kata yang harus selalu muncul dengan tercetak 
miring atau tercetak tebal. 
Dengan menjadikan kata-kata tersebut sebagai sebuah perintah \latex~tentu akan 
mempercepat dan mempermudah pengerjaan laporan. 


%-----------------------------------------------------------------------------%
\section{hype.indonesia.tex}
%-----------------------------------------------------------------------------%
Berkas ini berisi cara pemenggalan beberapa kata dalam bahasa Indonesia. 
\latex~memiliki algoritma untuk memenggal kata-kata sendiri, namun untuk 
beberapa kasus algoritma ini memenggal dengan cara yang salah. 
Untuk memperbaiki pemenggalan yang salah inilah cara pemenggalan yang benar 
ditulis dalam berkas hype.indonesia.tex.


%-----------------------------------------------------------------------------%
\section{Membuat Kutipan}
%-----------------------------------------------------------------------------%
Perintah yang dapat digunakan mengutip suatu artikel atau buku: 
\begin{itemize}
	\item  Untuk merujuk pada salah satu referensi yang ada, gunakan perintah \texttt{ \bslash cite}. Sebagai contoh: \texttt{ \bslash cite\{Brooks\}}, \texttt{\bslash cite\{Nance\}}, dan \texttt{\bslash cite\{Wang\}}
	yang akan akan memunculkan  \cite{Brooks}, \cite{Nance} dan \cite{Wang}  
	\item Untuk merujuk beberapa artikel pada akhir kalimat dan dengan tanda kurung, dapat menggunakan perintah \texttt{\bslash cite\{pengarang1,pengarang2\}}. Contoh: \texttt{\bslash citep\{Ross2005, Gensowski2013, Hermanto2009, wbid2010, Takyi2012, Fulan2014\}} memunculkan
	\citep{Ross2005, Gensowski2013, Hermanto2009, wbid2010, Takyi2012, Fulan2014}.
	\item Untuk menambahkan kata-kata dalam kutipan dalam tanda kurung, dapat menggunakan perintah \texttt{\bslash [kata] [kata] \{pengarang1\}}. Sebagai contoh:
	\texttt{\bslash citep[lihat][]\{Ross2005\}} akan menampilkan \citep[lihat][]{Ross2005}.
 	\item Daftar pustaka akan ditampilkan setelah Bab \babLima.
 	\item Anda bisa membuat model daftar referensi lain dengan menggunakan bibtex.
Untuk mempelajari bibtex lebih lanjut, silahkan buka 
\url{http://www.bibtex.org/Format}. 

\end{itemize}



%-----------------------------------------------------------------------------%
\section{Perintah Lain dalam Dokumen \latex~Ini}
\subsection{Mengubah Tampilan Teks}
%-----------------------------------------------------------------------------%
Beberapa perintah yang dapat digunakan untuk mengubah tampilan adalah: 
\begin{itemize}
	\item \texttt{\bslash f} \\
		Merupakan alias untuk perintah \texttt{\bslash textit}, contoh 
		\f{contoh hasil tulisan}.
	\item \texttt{\bslash bi} \\
		\bi{Contoh hasil tulisan}.
	\item \texttt{\bslash bo} \\
		\bo{Contoh hasil tulisan}.
	\item \texttt{\bslash m} \\
		\m{Contoh hasil tulisan}.
	\item \texttt{\bslash mc} \\
		\mc{Contoh hasil tulisan}.
	\item \texttt{\bslash code} \\ 
		\code{Contoh hasil tulisan}.
\end{itemize}


%-----------------------------------------------------------------------------%
\subsection{Memberikan Catatan}
%-----------------------------------------------------------------------------%
Ada dua perintah untuk memberikan catatan penulisan dalam dokumen yang Anda 
kerjakan, yaitu: 
\begin{itemize}
	\item \texttt{\bslash todo} \\
		Contoh: \\ \todo{Contoh bentuk todo.}
	\item \texttt{\bslash todoCite} \\ 
		Contoh: \todoCite
\end{itemize}


%-----------------------------------------------------------------------------%
\subsection{Menambah Isi Daftar Isi}
%-----------------------------------------------------------------------------%
Terkadang ada kebutuhan untuk memasukan kata-kata tertentu ke dalam Daftar Isi. 
Perintah \texttt{\bslash addChapter} dapat digunakan untuk judul bab dalam Daftar isi. 
Contohnya dapat dilihat pada berkas thesis.tex.


%-----------------------------------------------------------------------------%
\section{Memasukan PDF}
%-----------------------------------------------------------------------------%
Untuk memasukan PDF dapat menggunakan perintah \texttt{\bslash inpdf} yang menerima satu 
buah argumen. Argumen ini berisi nama berkas yang akan digabungkan dalam 
laporan. PDF yang dimasukan dengan cara ini akan memiliki \textit{header} dan \textit{footer} 
seperti pada halaman lainnya. 

\inpdf{include}

Cara lain untuk memasukan PDF adalah dengan menggunakan perintah \texttt{\bslash putpdf} 
dengan satu argumen yang berisi nama berkas pdf. Berbeda dengan perintah 
sebelumnya, PDF yang dimasukan dengan cara ini tidak akan memiliki footer atau 
header seperti pada halaman lainnya. 

\putpdf{include}


%-----------------------------------------------------------------------------%
\section{Membuat Perintah Baru}
%-----------------------------------------------------------------------------%
Ada dua perintah yang dapat digunakan untuk membuat perintah baru, yaitu: 
\begin{itemize}
	\item \texttt{\bslash Var} \\
		Digunakan untuk membuat perintah baru, namun setiap kata yang diberikan
		akan diproses dahulu menjadi huruf kapital. 
		Contoh jika perintahnya adalah \texttt{\bslash Var\{adalah\}} maka ketika 
		perintah \texttt{\bslash Var} dipanggil, yang akan muncul adalah ADALAH. 
	\item \texttt{\bslash var} \\
		Digunakan untuk membuat perintah atau baru. 
\end{itemize}









