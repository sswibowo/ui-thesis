%-----------------------------------------------------------------------------%
\chapter{\babSatu}
%-----------------------------------------------------------------------------%
\todo{tambahkan kata-kata pengantar bab 1 disini}

%-----------------------------------------------------------------------------%
\section{Latar Belakang}
%-----------------------------------------------------------------------------%
Standar penulisan ilmiah Universitas Indonesia dapat dilihat di \url{http://www.ui.ac.id/download/files/Pedoman-TA-UI\% 20-SK-Rektor-2008.pdf}. Contoh panduan penulisan ilmiah dapat dilihat di \url{https://scele.ui.ac.id/berkas_kolaborasi/konten/mpktb_
2014genap/BI.pdf}


Hindari kata-kata tidak baku sebagai contoh seperti: dimana, aktifitas, analisa, antri, azas, apotik, detil, Pebruari, hakekat, ijasah, handal, hipotesa, kwalitatif, subyektif. Daftar kata baku dalam bahasa Indonesia dapat dilihat di \url{http://kbbi.web.id}.


%-----------------------------------------------------------------------------%
\section{Permasalahan}
%-----------------------------------------------------------------------------%
Pada bagian ini akan dijelaskan mengenai definisi permasalahan 
yang \saya~hadapi dan ingin diselesaikan serta asumsi dan batasan 
yang digunakan dalam menyelesaikannya.


%-----------------------------------------------------------------------------%
\subsection{Definisi Permasalahan}
%-----------------------------------------------------------------------------%
\todo{Tuliskan permasalahan yang ingin diselesaikan. Bisa juga
	berbentuk pertanyaan}


%-----------------------------------------------------------------------------%
\subsection{Batasan Permasalahan}
%-----------------------------------------------------------------------------%
\todo{Umumnya ada asumsi atau batasan yang digunakan untuk 
	menjawab pertanyaan-pertanyaan penelitian diatas.}


%-----------------------------------------------------------------------------%
\section{Tujuan}
%-----------------------------------------------------------------------------%
\todo{Tuliskan tujuan penelitian.}


%-----------------------------------------------------------------------------%
\section{Posisi Penelitian}
%-----------------------------------------------------------------------------%
\todo{Posisi penelitian Anda jika dilihat secara bersamaan dengan 
	peneliti-peneliti lainnya. Akan lebih baik lagi jika ikut menyertakan 
	diagram yang menjelaskan hubungan dan keterkaitan antar 
	penelitian-penelitian sebelumnya}


%-----------------------------------------------------------------------------%
\section{Metodologi Penelitian}
%-----------------------------------------------------------------------------%
\todo{Tuliskan metodologi penelitian yang digunakan.}


%-----------------------------------------------------------------------------%
\section{Sistematika Penulisan}
%-----------------------------------------------------------------------------%
Sistematika penulisan laporan adalah sebagai berikut:
\begin{itemize}
	\item Bab 1 \babSatu \\
	Berikan penjelasan.
	\item Bab 2 \babDua \\
	Berikan penjelasan.	
	\item Bab 3 \babTiga \\
	Berikan penjelasan.
	\item Bab 4 \babEmpat \\
	Berikan penjelasan.
	\item Bab 5 \babLima \\
	Berikan penjelasan.
\end{itemize}

\todo{Tambahkan penjelasan singkat mengenai isi masing-masing bab.}

